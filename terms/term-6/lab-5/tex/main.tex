\documentclass[a4paper, 14pt]{extarticle}
\usepackage[russian]{babel}
\usepackage[T1]{fontenc}
\usepackage{fontspec}
\usepackage{indentfirst}
\usepackage{enumitem}
\usepackage{graphicx}
\usepackage[
  left=20mm,
  right=10mm,
  top=20mm,
  bottom=20mm
]{geometry}
\usepackage{parskip}
\usepackage{titlesec}
\usepackage{xurl}
\usepackage{hyperref}
\usepackage{float}
\usepackage[
  figurename=Рисунок,
  labelsep=endash,
]{caption}
\usepackage[outputdir=build, newfloat]{minted}
\usepackage[all]{hypcap}

\hypersetup{
  colorlinks=true,
  linkcolor=black,
  filecolor=blue,
  urlcolor=blue,
}

\renewcommand*{\labelitemi}{---}
\setmainfont{Times New Roman}
\setmonofont{JetBrains Mono}[
  SizeFeatures={Size=11},
]

\newenvironment{code}{\captionsetup{type=listing}}{}
\SetupFloatingEnvironment{listing}{name=Листинг}

\setminted{
  fontsize=\footnotesize,
}

\setlength{\parskip}{6pt}

\setlength{\parindent}{1cm}
\setlist[itemize]{itemsep=0em,topsep=0em,parsep=0em,partopsep=0em,leftmargin=2.0cm,wide}
\setlist[enumerate]{itemsep=0em,topsep=0em,parsep=0em,partopsep=0em,leftmargin=2.0cm,wide}

\renewcommand{\thesection}{\arabic{section}.}
\renewcommand{\thesubsection}{\thesection\arabic{subsection}.}
\renewcommand{\thesubsubsection}{\thesubsection\arabic{subsubsection}.}

\titleformat{\section}{\normalfont\bfseries}{\thesection}{0.5em}{}
\titleformat{\subsection}{\normalfont\bfseries}{\thesubsection}{0.5em}{}

\titleformat*{\section}{\normalfont\bfseries}
\titleformat*{\subsection}{\normalfont\bfseries}

\linespread{1.5}
\renewcommand{\baselinestretch}{1.5}

\begin{document}

\begin{titlepage}
  \vspace{0pt plus2fill}
  \noindent

  \vspace{0pt plus6fill}
  \begin{center}
    Санкт-Петербургский национальный исследовательский университет
    информационных технологий, механики и оптики

    \vspace{0pt plus3fill}

    Факультет инфокоммуникационных технологий

    Направление подготовки 11.03.02

    \vspace{0pt plus2fill}

    Лабораторная работа №5

    <<Работа с XML-схемами>>

  \end{center}

  \vspace{0pt plus6fill}
  \begin{flushright}
    Выполнил: \\
    Швалов Даниил Андреевич

    Группа: К33211

    Проверила: \\
    Марченко Елена Вадимовна
  \end{flushright}

  \vspace{0pt plus5fill}
  \begin{center}
    Санкт-Петербург

    2024
  \end{center}
\end{titlepage}

\section{Введение}

\textbf{Цель работы}: создание XML-схем для проверки различных XML-документов.

\section{Ход работы}

\subsection*{Упражнение №1. Создание простой XML-схемы}

В данном упражнении необходимо создать XML-схему для
\foreignlanguage{english}{lessons.xml} из прошлых лабораторных работ.

На рисунке \ref{fig:task-1:lessons_XML.xml} представлена получившаяся XML-схема.
В ней представлены в те же элементы и их поля, что были в DTD-схеме. Для полей
элемента \textit{\foreignlanguage{english}{lesson}} использовались такие типы
данных как \textit{\foreignlanguage{english}{string}} и
\textit{\foreignlanguage{english}{positiveInteger}}.

\vspace*{2em}

\bgroup
\inputminted{xml}{../code/task-1/lessons_XML.xml}
\captionof{figure}{XML-схема для документа lessons.xml}
\label{fig:task-1:lessons_XML.xml}
\egroup

\subsection*{Упражнение №2. Работа с простыми типами данных XML-схемы}

В данном упражнении необходимо создать XML-схему для
\foreignlanguage{english}{resume.xml}, созданного в прошлых лабораторных
работах. В данной XML-схеме необходимо, чтобы:
\begin{itemize}
  \item номер телефона проверялся с помощью регулярного выражения и был вида
  \#\#\#-\#\#-\#\# (т.е. последовательность <<три символа дефис два символа
  дефис два символа>>);
  \item дата рождения могла быть не раньше 1 января 1947 года и не позже 1
  января 1992 года;
  \item элемент <<Семейное положение>> ограничивался значениями <<женат>>, <<не
  женат>>, <<замужем>>, <<не замужем>>;
  \item элемент <<Образование>> ограничивался значениями <<высшее>>,
  <<среднее>>;
  \item возраст ограничивался значениями от 20 до 65 лет.
\end{itemize}

Согласно упражнению, для номера телефона, даты рождения, семейного положения,
образования и возраста были созданы собственные типы, проверяющие корректность
данных (рисунок \ref{fig:task-2:resume_XML.xml}). Для номера телефона
используется регулярное выражение, которое проверяет, что номер соответствует
формату <<\#\#\#-\#\#-\#\#>>. Для даты рождения используется стандартный тип
\textit{\foreignlanguage{english}{date}} с ограничениями по минимальному и
максимальному значению. Для семейного положения и образования используется
тип-перечисление. Для возраста был создан тип, который принимает неотрицательное
число и проверяет, что оно находится в заданном диапазоне.

\inputminted{xml}{../code/task-2/resume_XML.xml}
\begin{figure}[H]
  \caption{XML-схема для документа resume.xml}
  \label{fig:task-2:resume_XML.xml}
\end{figure}

\subsection*{Упражнение №3. Работа с XML-схемой}

В данном упражнении необходимо создать XML-схему для документа, созданного в
третьем упражнении прошлых лабораторных работ.

В XML-схеме, представленной на рисунке \ref{fig:task-3:books_XML.xml}, были
созданы отдельные типы данных для ISBN, языка, возрастного ограничения. Так,
например, для ISBN производится проверка соответствия введенного значения
формату <<\#\#\#-\#-\#\#-\#\#\#\#\#\#-\#>> с помощью регулярного выражения. Для
проверки корректности языка и возрастного ограничения используется перечисление.
Также были созданы типы для всех остальных элементов, которые раньше были
представлены в DTD-схеме.

\inputminted{xml}{../code/task-3/books_XML.xml}
\begin{figure}[H]
  \caption{XML-схема для документа books.xml}
  \label{fig:task-3:books_XML.xml}
\end{figure}

\section{Вывод}

В ходе выполнения лабораторной работы были созданы XML-схемы для проверки
корректности XML-документов.

\end{document}
