\documentclass[a4paper, 14pt]{extarticle}
\usepackage[russian]{babel}
\usepackage[T1]{fontenc}
\usepackage{fontspec}
\usepackage{indentfirst}
\usepackage{enumitem}
\usepackage{graphicx}
\usepackage[
  left=20mm,
  right=10mm,
  top=20mm,
  bottom=20mm
]{geometry}
\usepackage{parskip}
\usepackage{titlesec}
\usepackage{xurl}
\usepackage{hyperref}
\usepackage{float}
\usepackage[
  figurename=Рисунок,
  labelsep=endash,
]{caption}
\usepackage[outputdir=build, newfloat]{minted}

\hypersetup{
  colorlinks=true,
  linkcolor=black,
  filecolor=blue,
  urlcolor=blue,
}

\renewcommand*{\labelitemi}{---}
\setmainfont{Times New Roman}
\setmonofont{JetBrains Mono}[
  SizeFeatures={Size=11},
]

\newenvironment{code}{\captionsetup{type=listing}}{}
\SetupFloatingEnvironment{listing}{name=Листинг}

\setminted{
  fontsize=\footnotesize,
}

\setlength{\parskip}{6pt}

\setlength{\parindent}{1cm}
\setlist[itemize]{itemsep=0em,topsep=0em,parsep=0em,partopsep=0em,leftmargin=2.0cm,wide}
\setlist[enumerate]{itemsep=0em,topsep=0em,parsep=0em,partopsep=0em,leftmargin=2.0cm,wide}

\renewcommand{\thesection}{\arabic{section}.}
\renewcommand{\thesubsection}{\thesection\arabic{subsection}.}
\renewcommand{\thesubsubsection}{\thesubsection\arabic{subsubsection}.}

\titleformat{\section}{\normalfont\bfseries}{\thesection}{0.5em}{}
\titleformat{\subsection}{\normalfont\bfseries}{\thesubsection}{0.5em}{}

\titleformat*{\section}{\normalfont\bfseries}
\titleformat*{\subsection}{\normalfont\bfseries}

\linespread{1.5}
\renewcommand{\baselinestretch}{1.5}

\begin{document}

\begin{titlepage}
  \vspace{0pt plus2fill}
  \noindent

  \vspace{0pt plus6fill}
  \begin{center}
    Санкт-Петербургский национальный исследовательский университет
    информационных технологий, механики и оптики

    \vspace{0pt plus3fill}

    Факультет инфокоммуникационных технологий

    Направление подготовки 11.03.02

    \vspace{0pt plus2fill}

    Лабораторная работа №4

    <<Работа с DTD схемами документов>>

  \end{center}

  \vspace{0pt plus6fill}
  \begin{flushright}
    Выполнил: \\
    Швалов Даниил Андреевич

    Группа: К33211

    Проверила: \\
    Марченко Елена Вадимовна
  \end{flushright}

  \vspace{0pt plus5fill}
  \begin{center}
    Санкт-Петербург

    2024
  \end{center}
\end{titlepage}

\section{Введение}

\textbf{Цель работы}: создание DTD-схемам для проверки корректности различных
XML-документов.

\section{Ход работы}

\subsection*{Упражнение №1. Создание DTD схемы документа}

В этом упражнении необходимо создать DTD-схему для документа
\textit{\foreignlanguage{english}{lessons.xml}} из прошлой лабораторной работы.
Содержимое документа \textit{\foreignlanguage{english}{lessons.xml}}
представлено на рисунке \ref{fig:task-1:lessons.xml}. В данном документе
содержатся следующие сущности:
\begin{itemize}
  \item \textbf{lessons}: сущность, представляющая список дисциплин.
  \item \textbf{lesson}: сущность, представляющая конкретную дисциплину.
  Содержит название дисциплины, фамилию преподавателя, номер семестра,
  количество академических часов, тип контроля и формат проведения. Также
  обладает уникальным идентификатором.
  \item \textbf{CourseName}: сущность, представляющая название дисциплины.
  \item \textbf{ProfessorSurname}: сущность, представляющая имя преподавателя,
  который проводит данную дисциплину.
  \item \textbf{Semester}: сущность, представляющая номер семестра, в котором
  проводится дисциплина.
  \item \textbf{AcademicHours}: сущность, представляющая количество
  академических часов данной дисциплины.
  \item \textbf{ControlType}: сущность, представляющая тип контроля (например,
  экзамен или зачёт).
  \item \textbf{Format}: сущность, представляющая формат проведения дисциплины
  (например, очный или дистанционный).
\end{itemize}

\begin{figure}[H]
  \centering
  \inputminted{xml}{../code/task-1/lessons.xml}
  \caption{Содержимое документа \textit{lessons.xml}}
  \label{fig:task-1:lessons.xml}
\end{figure}

Исходя из ранее приведенного описания сущностей была сделана DTD-схема,
приведенная на рисунке \ref{fig:task-1:lessons_DTD.dtd}. Для идентификаторов
дисциплин используется тип \textit{ID}, который проверяет уникальность
идентификатора. Для конченых значений, таких как название дисциплины,
используется тип \textit{PCDATA}, который позволяет хранить в себе данные,
которые будут интерпретированы как разметка XML.

\begin{figure}[H]
  \centering
  \inputminted{dtd}{../code/task-1/lessons_DTD.dtd}
  \caption{Содержимое схемы \textit{lessons\_DTD.dtd}}
  \label{fig:task-1:lessons_DTD.dtd}
\end{figure}

\subsection*{Упражнение №2. Работа с DTD-схемой}

В данном упражнении необходимо для существующей DTD-схемы создать XML-документ,
который будет удовлетворять этой схеме. В качестве DTD-схемы был дан файл,
показанный на рисунке \ref{fig:task-2:library_DTD.dtd}. В нем содержатся
следующие сущности:
\begin{itemize}
  \item \textbf{library}: представляет собой библиотеку, хранит информацию о
  книгах и авторах.
  \item \textbf{book\_catalog}: представляет собой каталог книг, хранит
  информацию о всех книгах.
  \item \textbf{book}: представляет собой книгу, хранит информацию об авторах
  книги, ее заголовок, издательство и аннотацию к книге. Имеет такие атрибуты,
  как идентификатор, ISBN, год издания и тип (перевод или оригинал).
  \item \textbf{authors}: представляет собой авторов книги.
  \item \textbf{author}: представляет собой конкретного автора книги.
  \item \textbf{title}: представляет собой заголовок книги.
  \item \textbf{publishing}: представляет собой издательство.
  \item \textbf{annotation}: представляет собой аннотацию к книге.
  \item \textbf{author\_catalog}: представляет собой каталог авторов, хранит
  информацию о книгах авторов.
  \item \textbf{author\_book}: представляет собой книгу автора. Имеет ссылку на
  идентификатор книги.
\end{itemize}

\begin{figure}[H]
  \centering
  \inputminted{dtd}{../code/task-2/library_DTD.dtd}
  \caption{Содержимое схемы \textit{library\_DTD.dtd}}
  \label{fig:task-2:library_DTD.dtd}
\end{figure}

Исходя из ранее описанных атрибутов, был составлен XML-документ, приведенный на
рисунке \ref{fig:task-2:library.xml}. В нем приведены три книги, каждая из
которых по-разному использует различные сущности. Так, первая сущность содержит
всевозможные атрибуты и поля. Вторая сущность показывает, что книга может иметь
несколько авторов. Третья сущность практически не обладает полями и
демонстрирует работу необязательных полей. Также в документе созданы две ссылки
авторов на конкретные книги.

\inputminted{xml}{../code/task-2/library.xml}
\begin{figure}[H]
  \centering
  \caption{Содержимое документа \textit{library.xml}}
  \label{fig:task-2:library.xml}
\end{figure}

\subsection*{Упражнение №3. Создание DTD-схемы}

В предыдущей лабораторной работе был создан XML-документ
\textit{\foreignlanguage{english}{books.xml}}, содержимое которого представлено
на рисунке \ref{fig:task-3:books.xml}. В нем содержатся следующие сущности:
\begin{itemize}
  \item \textbf{books}: сущность, представляющая список всех книг.
  \item \textbf{book}: сущность, представляющая конкретную книгу. Содержит
  заголовок, аннотацию, автора, жанр, количество страниц, язык издания,
  издательство, год издания, ISBN и возрастное ограничение. Также имеет
  уникальный идентификатор.
  \item \textbf{title}: сущность, представляющая заголовок книги.
  \item \textbf{summary}: сущность, представляющая аннотацию к книге.
  \item \textbf{authors}: сущность, представляющая авторов книги.
  \item \textbf{author}: сущность, представляющая конкретного автора книги.
  \item \textbf{genres}: сущность, представляющая жанры книги.
  \item \textbf{genre}: сущность, представляющая конкретный жанр книги.
  \item \textbf{pages}: сущность, представляющая количество страниц.
  \item \textbf{language}: сущность, представляющая язык, на котором издана
  книга.
  \item \textbf{publisher}: сущность, представляющая издательство.
  \item \textbf{publication\_year}: сущность, представляющая год издания книги.
  \item \textbf{isbn}: сущность, представляющая ISBN книги.
  \item \textbf{age\_limit}: сущность, представляющая возрастное ограничение
  книги.
\end{itemize}

\inputminted{xml}{../code/task-3/books.xml}
\begin{figure}[H]
  \centering
  \caption{Содержимое документа \textit{books.xml}}
  \label{fig:task-3:books.xml}
\end{figure}

Исходя из выше описанных сущностей, была создана DTD-схема, представленная на
рисунке \ref{fig:task-3:books_DTD.dtd}.

\begin{figure}[H]
  \centering
  \inputminted{dtd}{../code/task-3/books_DTD.dtd}
  \caption{Содержимое схемы \textit{books\_DTD.dtd}}
  \label{fig:task-3:books_DTD.dtd}
\end{figure}

\section{Вывод}

В ходе выполнения лабораторной работы были созданы DTD-схемы для проверки
корректности XML-документов.

\end{document}
