\documentclass[a4paper, 14pt]{extarticle}
\usepackage[russian]{babel}
\usepackage[T1]{fontenc}
\usepackage{fontspec}
\usepackage{indentfirst}
\usepackage{enumitem}
\usepackage{graphicx}
\usepackage[
  left=20mm,
  right=10mm,
  top=20mm,
  bottom=20mm
]{geometry}
\usepackage{parskip}
\usepackage{titlesec}
\usepackage{xurl}
\usepackage{hyperref}
\usepackage{float}
\usepackage[
  figurename=Рисунок,
  labelsep=endash,
]{caption}
\usepackage[outputdir=build, newfloat]{minted}
\usepackage[all]{hypcap}

\hypersetup{
  colorlinks=true,
  linkcolor=black,
  filecolor=blue,
  urlcolor=blue,
}

\renewcommand*{\labelitemi}{---}
\setmainfont{Times New Roman}
\setmonofont{JetBrains Mono}[
  SizeFeatures={Size=11},
]

\newenvironment{code}{\captionsetup{type=listing}}{}
\SetupFloatingEnvironment{listing}{name=Листинг}

\setminted{
  fontsize=\footnotesize,
}

\setlength{\parskip}{6pt}

\setlength{\parindent}{1cm}
\setlist[itemize]{itemsep=0em,topsep=0em,parsep=0em,partopsep=0em,leftmargin=2.0cm,wide}
\setlist[enumerate]{itemsep=0em,topsep=0em,parsep=0em,partopsep=0em,leftmargin=2.0cm,wide}

\renewcommand{\thesection}{\arabic{section}.}
\renewcommand{\thesubsection}{\thesection\arabic{subsection}.}
\renewcommand{\thesubsubsection}{\thesubsection\arabic{subsubsection}.}

\titleformat{\section}{\normalfont\bfseries}{\thesection}{0.5em}{}
\titleformat{\subsection}{\normalfont\bfseries}{\thesubsection}{0.5em}{}

\titleformat*{\section}{\normalfont\bfseries}
\titleformat*{\subsection}{\normalfont\bfseries}

\linespread{1.5}
\renewcommand{\baselinestretch}{1.5}

\begin{document}

\begin{titlepage}
  \vspace{0pt plus2fill}
  \noindent

  \vspace{0pt plus6fill}
  \begin{center}
    Санкт-Петербургский национальный исследовательский университет
    информационных технологий, механики и оптики

    \vspace{0pt plus3fill}

    Факультет инфокоммуникационных технологий

    Направление подготовки 11.03.02

    \vspace{0pt plus2fill}

    Лабораторная работа №6

    <<Введение в XSLT и работа с XPath выражениями>>

  \end{center}

  \vspace{0pt plus6fill}
  \begin{flushright}
    Выполнил: \\
    Швалов Даниил Андреевич

    Группа: К33211

    Проверила: \\
    Марченко Елена Вадимовна
  \end{flushright}

  \vspace{0pt plus5fill}
  \begin{center}
    Санкт-Петербург

    2024
  \end{center}
\end{titlepage}

\section{Введение}

\textbf{Цель работы}: создание XPath выражений для извлечения данных из
XML-документов.

\section{Ход работы}

\subsection*{Упражнение №1. Работа с XPath выражениями}

В данном упражнении необходимо написать XPath выражения для документа
\foreignlanguage{english}{lessons.xml}, которые выводят
\begin{enumerate}
  \item второе занятие;
  \item предпоследнее занятие;
  \item общее количество занятий;
  \item количество академический часов третьего занятия.
\end{enumerate}

Получившиеся XPath представлены на рисунке \ref{fig:task-1:less_template_2.xml}.
Второе занятие было получено с помощью предиката
\foreignlanguage{english}{\textit{position() = 2}}. Предпоследнее занятие было
получено похожим образом: из последнего занятия (с помощью предиката
\foreignlanguage{english}{\textit{last()}}) с помощью вычитания было получено
предпоследнее занятие. Количество занятий было получено с помощью предиката
\foreignlanguage{english}{\textit{count()}}. Количество академических часов
третьего занятия было получено с помощью предиката
\foreignlanguage{english}{\textit{position() = 3}} и выбором дочернего узла.

\bgroup
\inputminted{xml}{../code/task-1/less_template_2.xsl}
\captionof{figure}{XPath выражения для документа lessons.xml}
\label{fig:task-1:less_template_2.xml}
\egroup

\subsection*{Упражнение №2. XPath выражения}

В данном упражнении необходимо написать XPath выражения для документа
\foreignlanguage{english}{resume.xml}, которые выводят тех кандидатов, у
которых:
\begin{enumerate}
  \item возраст меньше 32 лет;
  \item возраст больше или равен 33 годам.
\end{enumerate}

Получившиеся XPath представлены на рисунке \ref{fig:task-2:resume_template.xml}.
Для получения кандидатов младше 32 лет использовался предикат
\textit{\foreignlanguage{english}{number(.) < 32}}, который проверяет, что
численное значение, хранящееся в узле \textit{Age}, меньше 32. Для получения
кандидатов старше 32 лет использовался предикат
\textit{\foreignlanguage{english}{number(.) >= 33}}, который похож на
предыдущий, за исключением \textit{>= 33}. Так как угловые скобочки
зарезервированы в XML, вместо \textit{<} и \textit{<} используются \textit{\&lt;}
и \textit{\&gt;} соответственно.

\bgroup
\inputminted{xml}{../code/task-2/resume_template.xsl}
\captionof{figure}{XPath выражения для документа resume.xml}
\label{fig:task-2:resume_template.xml}
\egroup

\section{Вывод}

В ходе выполнения лабораторной работы были созданы XPath выражения для
извлечения данных из XML-документов.

\end{document}
